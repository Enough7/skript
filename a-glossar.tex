\chapter{Glossar}
\section{Begriffserklärungen}
\begin{description}
	\item[Bildraum] Für eine Funktion $f\colon A \rightarrow B$ bezeichnet $\{b \in B \mid \exists a \in A: f(a) = b\}$ den Bildraum.
	\item[Diskreter Logarithmus] Bezeichne $\mathbbm{G} = \langle g \rangle$ eine endliche zyklische Gruppe mit Ordnung $N$. Dann gibt es für
	$\forall h \in \mathbbm{G} : \exists x \in \mathbbm{Z}_N :  g^x \equiv h$ und es bezeichnet $x = \log_g h$ den diskreten Logarithmus von $h$
	bezüglich $g$.
	\item[Forward Secrecy] Unter dem Begriff der forward secrecy versteht man eine Eigenschaft von Schlüsselaustauschprotokollen, die fordert,
	dass der Sitzungsschlüssel, mit dem die Nutzdaten der Verbindung gesichert sind, nicht von den privaten Schlüsseln der Kommunikationspartner
	abgeleitet werden kann. Sollte in Zukunft eine der Parteien kompromittiert werden, können die verschlüsselten Nutzdaten vom Angreifer nicht
	ausgelesen werden. Wird bereits mindestens eine der Parteien während der Kommunikation von einem Angreifer kontrolliert, bietet die forward
	secrecy offensichtlich keinen Schutz.
	\item[Gleichverteilung] Gilt für eine Verteilung $U$ über der Menge $M$, dass
	\begin{align*}
		\forall x \in M : \Pr [x \leftarrow U] = \frac{1}{\vert M \vert}\, ,
	\end{align*}
	heißt $U$ Gleichverteilung.
	\item[Gruppe] Es sei $M$ eine Menge und $\ast$ eine abgeschlossene Verknüpfung auf $M$. Dann heißt $(M, \ast)$ eine Gruppe, falls
	\begin{enumerate}
		\item das Assoziativgesetz gilt,
		\item ein neutrales Element $e_M \in M$ und
		\item $\forall x \in M : x^{-1} \in M$.
	\end{enumerate}
	\item[Gruppenordnung] Bezeichne $\mathbbm{G} = (M, \ast)$, dann heißt $\vert M \vert$ Gruppenordnung von $\mathbbm{G}$. Umgangssprachlich
	schreibt man auch $\vert \mathbbm{G} \vert$.
	\item[Heuristik] Eine Heuristik ist eine plausible, aber nicht bewiesene, Annahme über ein System.
	\item[Homomorphismus] Ein Homomorphismus bezeichnet eine strukturerhaltende Abbildung. Für ein homomorphes Verschlüsselungsverfahren $\enc$
	und zwei Nachrichten $M_1, M_2$ (die Elemente einer additiven Gruppe sind) sähe das \emph{beispielsweise} folgendermaßen aus:
	\begin{align*}
		\enc(M_1 + M_2) = \enc(M_1) \cdot \enc(M_2)
	\end{align*}
	\item[Kollision] Falls für eine (Hash-)Funktion $H\colon A \rightarrow B$
	\begin{align*}
		\exists x, x' \in A : x \neq x' \land H(x) = H(x') 
	\end{align*}
	gilt, spricht man von einer Kollision in $H$.
	\item[Kryptographische Hashfunktion] Eine kryptographische Hashfunktion ist eine Hashfunktion, die mindestens eine der folgenden Eigenschaften
	- Kollisionsresistenz, target collision resistance oder Einwegeigenschaft - besitzt. Dabei ist die Kollisionsresistenz der stärkste Begriff und impliziert
	die target collision resistance, aus welcher wiederum die Einwegeigenschaft folgt.
	\item[Kryptosystem] Ein System bestehend aus Verschlüsselungs- und dazugehörigem Entschlüsselungsalgorithmus.
	\item[Man-in-the-Middle-Angriff] Bezeichnet einen Angriff, bei dem sich der Angreifer logisch zwischen den beiden Kommunikationspartner befindet und,
	je nachdem ob passiv oder aktiv, die Verbindung abhören oder manipulieren kann. Dazu zählt auch das Einschleusen eigener Information.
	\item[Padding] Ein Mechanismus, um eine gewisse Menge an Daten auf eine vorgeschriebene (Block-)Länge aufzufüllen.
	\item[Permutation] Bezeichne $\{L_n\}$ die Menge geordneter Listen der Elemente $\{l_1, \dots, l_n\}$. Dann heißt $\phi\colon \{L_n\} \rightarrow \{L_n\}$ eine Permutation.
	\item[Prüfsumme] Ein Mechanismus zur (approximativen) Gewährleistung der Datenintegrität bei Datenübertragung und Datensicherung.
	\item[Replay-Angriff] Bei einer Replay-Angriff auf eine (Daten-)Verbindung zeichnet der Angreifer zunächst passiv gesendete Information auf, um
	sie im späteren Verlauf erneut einer der Parteien zu schicken.
	\item[Schlüsselzentrale] Eine Schlüsselzentrale bezeichnet eine abstrakte Einheit in einer Secret-Key- oder Public-Key-Infrastruktur, die für das
	Erstellen, Verwalten und Verteilen von Schlüsseln verantwortlich ist.
	\item[Semantik] Die ursprüngliche Wortbezeichnung beschreibt ein Teilgebiet der Linguistik, dass sich mit der Bedeutung von Zeichen oder Zeichenfolgen
	auseinandersetzt. Im informationstheoretisch-kryptographischen Kontext wird es gelegentlich auch synonym zu Information verwendet (Vgl. 
	\ref{def:semsec}).
	\item[Untergruppe] Bezeichne $\mathbbm{G} = (M, \ast)$ eine Gruppe. Dann bezeichnet $\mathbbm{H} = (M', \ast)$ eine Untergruppe von $\mathbbm{G}$,
	falls
	\begin{enumerate}
		\item $M' \subseteq M$,
		\item die Verknüpfung $\ast$ in $\mathbbm{H}$ abgeschlossen ist,
		\item das neutrale Element $e_M \in \mathbbm{H}$ und
		\item für alle $x \in \mathbbm{H} : x^{-1} \in \mathbbm{H}$.
	\end{enumerate}
	Umgangssprachlich schreibt man $\mathbbm{H} \subseteq \mathbbm{G}$.
	\item[Urbildraum] Für eine Funktion $f\colon A \rightarrow B$ bezeichnet $A$ den Urbildraum.
	\item[Zielraum] Für eine Funktion $f\colon A \rightarrow B$ bezeichnet $B$ den Zielraum.
\end{description}

\renewcommand{\arraystretch}{1.5}
\section{Mathematische Bezeichnungen}
\begin{tabularx}{\textwidth}{ p{.12\textwidth} | X }
	$\mathbbm{Z}^{\ast}_p$ & Zyklische multiplikative Gruppe ganzer Zahlen, die kleiner $p$ und koprim zu $p$ sind, das heißt $\{x : \ggT(x, p) = 1\}$\\
	$\mathbbm{Z}_N$ & Zyklische additive Gruppe ganzer Zahlen modulo $N$, das heißt ${\{0, \dots, N-1\}}$\\
	$\mathbbm{F}^{\ast}_q$ & Multiplikative Gruppe des dazugehörigen Galois-Körpers $\mathbbm{F}_q$
\end{tabularx}

\section{Notationsformalismus}
\begin{tabularx}{\textwidth}{ p{.12\textwidth} | X }
	$\A^{\B}$ & Die Turing-Maschine $\A$ hat Orakelzugriff auf Turing-Maschine $\B$\\
	$A \mid B$ & Der Ausdruck $A$ teilt Ausdruck $B$ ohne Rest, d.h. $\exists k \in \mathbbm{Z} : k \cdot A = B$\\
	$x \leftarrow D$ & Der Variable $x$ wird (probabilistisch) ein Wert der Wahrscheinlichkeitsverteilung $D$ zugewiesen\\
	$x \randUnif M$ & Der Variable $x$ wird zufällig gleichverteilt ein Wert der Menge $M$ zugewiesen\\
	$M_1 \concat M_2$ & Bezeichnet die Konkatenation zweier Bit-Strings $M_1$ und $M_2$\\
	$\calP(M)$ & Bezeichnet die Potenzmenge der Menge $M$, d.h. $\{U : U \subseteq M\}$\\
	$\bot$ & Der Bottom Type bedeutet, dass kein Wert zurückgegeben wird und wird in diesem Skript als Fehlersymbol verwendet\\
	$O(f(n))$ & Bezeichnet die Menge
	\begin{align*}
		\{g(n) : \exists c \in \mathbbm{R}^{+}, n_0 \in \mathbbm{N} : \forall n \geq n_0 : 0 \leq g(n) \leq c \cdot f(n)\}
	\end{align*}\\
	$\Omega(f(n))$ & Bezeichnet die Menge
	\begin{align*}
		\{g(n) : \exists c \in \mathbbm{R}^{+}, n_0 \in \mathbbm{N} : \forall n \geq n_0 : 0 \leq c \cdot f(n) \leq g(n)\}
	\end{align*}\\
	$\Theta(f(n))$ & Bezeichnet die Menge
	\begin{align*}
		\{g(n) : g(n) \in O(f(n)) \land g(n) \in \Omega(f(n))\}
	\end{align*}\\
	%$Adv^{cr}_{H,\A}(k)$ & Angreifer\\
	%$Adv^{ow}_{H,\A}(k)$ & dsa\\
	%$Adv^{tcr}_{H,\A}(k)$ & dsa
\end{tabularx}

\section{Komplexitätsklassen}
\begin{tabularx}{\textwidth}{ p{.12\textwidth} | X }
	$P$ & $P$ ist die Menge der Sprachen $L$, für die es eine deterministische Turing-Maschine gibt, die in höchstens $p(\vert x \vert)$-Schritten entscheiden kann,
	ob ${x \in L}$, wobei $p$ ein beliebiges Polynom ist\\
	$NP$ & $NP$ ist die Menge der Sprachen $L$, für die es eine nichtdeterministische Turing-Maschine, gibt, die, falls ${x \in L}$, $x$ in höchstens
	$p(\vert x \vert)$-Schritten akzeptiert, wobei $p$ ein beliebiges Polynom ist\\
	$NPC$ & $NPC$ ist die Menge der Sprachen $L \in NP$, für die zusätzlich gilt: ${\forall L' \in NP : L' \leq^{TM} L}$, d.h. es existiert eine Turing-Maschine TM,
	die $L'$ auf $L$ in Polynomialzeit reduziert (Alternativ: $NP\text{-complete}$, $NP\text{-vollständig}$)\\
\end{tabularx}


%Unterscheider
%Meet-in-the-Middle
%binäre Suche
%Reduktionsfunktion
%hardcore-Bit