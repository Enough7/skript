\section{Abkürzungen}
\begin{tabular}{p{.1\textwidth} | p{.3\textwidth} | p{.6\textwidth}}
  CBC         & Cipher Block Chaining Mode                                 & Betriebsmodus f. Blockchiffren, siehe Kap. \ref{ssec:cbc}.       \\ \hline
  CDH-Annahme & Computational-Diffie-Hellman-Annahme                       & Annahme, dass es schwer ist, für zwei Gruppenelemente $g^a, g^b$                                
                                              das Gruppenelement $g^{ab}$ zu berechnen. Siehe Kap. \ref{sec:ddh-key-exchange}. \\ \hline
  CTR         & Counter Mode                                               & Betriebsmodus f. Blockchiffren, siehe Kap. \ref{sssec:ctr}.       \\ \hline
  DDH-Annahme & Decisional-Diffie-Hellman-Annahme                          & Annahme, dass es schwer ist, für drei Gruppenelemente                                           
                                  $g^a, g^b, g^c$ herrauszufinden, ob $g^c= g^{ab}$ ist. Siehe Def. \ref{def:ddh}. \\ \hline
  DLOG        & discrete logarithm                                         & Ziehen von Logarithmen in Gruppen.                               \\ \hline
  ECB         & Electronic Codebook Mode                                   & Betriebsmodus f. Blockchiffren, siehe Kap. \ref{sssec:ecb}.       \\ \hline
  EUF-CMA     & Existential Unforgeability - adaptiv Chosen Message Attack & Sicherheitsbegriff für Authentifikationsverfahren. Siehe Kap.                                   
                                           \ref{ch:symauth:sicherheit}                                      \\ \hline
  GCM         & Galois Counter Mode                                        & Betriebsmodus f. Blockchiffren  siehe Kap. \ref{sssec:gcm}.       \\ \hline
  HMAC        & Keyed-Hash Message Authentication Code                     & EUF-CMA-sichere MAC, die mit einer Merkle-Dåmgard-Konstruktion generiert wird.                  
                                                            Siehe Kap. \ref{ssec:hmac}.                                      \\ \hline
  negl.       & negliable                                                  & dt. \textit{vernachlässigbar}. Eine Funktion heißt negl., 
                                                                             wenn sie schneller fällt als der Kehrwert jeden Polynoms. Siehe Def. \ref{def:negl}.\\ \hline

  IND-CCA     & indistinguishability under chosen-ciphertext attacks       & Sicherheitsbegriff für Verschlüsselungsverfahren, siehe Kap. \ref{sec:ind-cca}. \\ \hline
  IND-CPA     & indistinguishability under chosen-plaintext attacks        & Sicherheitsbegriff für Verschlüsselungsverfahren, siehe Kap. \ref{sec:ind-cpa}. \\ \hline
  MAC         & Message Authentication Code                                & symmetrisches Verfahren, um die Authentizität einer Nachricht                                   
                                           sicherzustellen. Siehe Kap. \ref{ch:symauth:macs}.               \\ \hline
  PPT         & probabilistic polynomial time                              & Laufzeit von Algorithmen. Ein Algorithmus ist PPT, wenn er mit polynomiellem Aufwand und        
                                                                      einer  Fehlerwahrscheinlichkeit kleiner als $0,5$. Siehe Kap. \ref{sec:secparam}. \\ \hline 
  PRF         & Pseudorandomisierte Funktion                               & Funktion, für die es schwer ist, sie von echtem Zufall zu                                       
                                       unterscheiden. Siehe Kapitel \ref{ssec:prf}                      \\ \hline
  PSS         & Probabilistic Signature Scheme                             & gepaddete Variante von RSA. Siehe Kap. \ref{sec:rsa}             \\ \hline
  RSA         & Rivest, Shamir, Adleman                                    & Public-Key Verfahren für Verschlüsselung und Signaturen          \\ \hline
  RSA-OAEP    & RSE optimal asymmetric encryption padding                  & Gepappete, sicherere Variante von RSA. Siehe Kap. \ref{ssec:sicheres-rsa}  \\ \hline
  SHA         & Secure Hash Algorithm                                      & Standards für Familien von Hash-Verfahren                        \\ \hline
  TLS         & Tranport Layer Security                                    & Protokoll für verschlüsselte Verbindungen auf Transportebene.                                   
                                                                             Siehe Kap. \ref{sec:keyexchange:tls}.                            \\ \hline
\end{tabular}
%%% Local Variables:
%%% mode: latex
%%% TeX-master: "skript"
%%% End:
